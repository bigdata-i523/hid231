\documentclass[sigconf]{acmart}

\input{format/i523}

\begin{document}
\title{Using Big Data to Battle Air Pollution}

\author{Karthik Vegi}
\affiliation{%
  \institution{Indiana University Bloomington}
  \streetaddress{2619 East 2nd Street, Apt 11}
  \city{Bloomington, IN 47401} 
  \country{USA}}
\email{kvegi@iu.com}


% The default list of authors is too long for headers}
\renewcommand{\shortauthors}{kvegi}

\begin{abstract}
We have come a long way from the stone age to build large scale industries, big cities, bullet trains, and a booming automobile industry. Technological and industrial advances are making our cities smarter by the day and yet a nagging side-effect is air pollution. Air pollution is not only creating local health hazards like respiratory and heart problems, but also directly leading to an increase in temperatures and contributing to global warming. We show how the advances in {\em Big Data}, {\em Cloud Computing}, and {\em Internet Of Devices} can be used to combat air pollution.
\end{abstract}

\keywords{i523, hid231, big data, environment, air pollution, global warming}

\maketitle

\section{Introduction}
Air pollution is no longer a local problem. It is a global environmental issue which involves individual countries to come together and device measures to combat it \cite{www-ral}. It it causing about 3.7 million premature deaths worldwide from cardiovascular and respiratory diseases and also ruins the crops that feed the world \cite{www-ral}. Air pollution also has a direct effect on a number of environmental issues like global warming, depletion of ozone layer, acid rains and impacts wild-life \cite{www-ral}. \\
Back in the year 1990, the job of a typical air quality scientist was to develop atmospheric dispersion models to evaluate the air pollution caused by industries and make sure that it is within the permissible level suggested by the {\em Environmental Protection Agency} \cite{www-ibm1}. These models gather historic data of many years from airports and weather balloons to predict the pollution with the help of meteorology theory \cite{www-ibm1}. Although the methods used to derive the values were good enough, the limitations with respect to the technology posed a real challenge which took weeks to run the simulations only to be cut-off in the middle due to power and storage issues \cite{www-ibm1}. The data processing engine was built on Sun-Solaris workstations with tapes handling the data storage \cite{www-ibm1}. The work-stations set up in major points in the country would communicate using a very slow network connection \cite{www-ibm1}. The data processing would be done locally and later written to all the servers which would then be split and distributed among many machines and consolidated in the end \cite{www-ibm1}. ``If only we had that much more data and that much more ability to handle it, we could iterate through the model at a much finer scale. Real-time data processing remained a pipe dream'' \cite{www-ibm1}. \\

\section{Air Pollution as a Big Data Problem}
The advent of {\em Big Data} and the technological advances changed the way the data is ingested and analyzed \cite{www-ibm1}. The network speeds have increased, wide range of sensors are available to collect data with a lot of precision which would feed the high speed data processing systems. Batch processing has become easier with {\em Hadoop} and {\em Map-Reduce}. The storage mechanisms have become cheaper and more disaster proof. \\
{\em IBM} is helping Beijing combat air pollution by analyzing huge amounts of data using a data analysis platform {\em Green Horizons} \cite{www-huff}. IBM has signed up partnerships with different cities in China and India to deploy {\em Big Data Analytics}, {\em Machine Learning} and {\em Internet Of Things} to improve traffic, keep a check on the pollution from industrial machines and other pollution causing agents \cite{www-huff}. {\em IBM} will deploy sensors in various place to collect data in real-time and analyze previous weather forecasts, and build improved iterative models over time \cite{www-huff}. The system continuously streams data form the sensors and improves the forecast by learning over time using {\em Machine Learning} algorithms \cite{www-huff}. \\

\begin{figure}
\includegraphics[width=0.7\textwidth]{images/fig1.png}
\caption{Green Horizons air quality management for Beijing \cite{www-huff}}
\end{figure}

{\em IBM} is collaborating with the United Nations to push the use of technological advances by every country for the common good of the world \cite{www-huff}. More and more cities and countries are opening air quality data to public where you can get reports in real time \cite{www-ferro}. The {\em BreezoMeter} is the first mobile application that provides real-time information of the street's air quality information using geo-location maps \cite{www-ferro}. {\em Copernicus} is another monitoring service that ingests data from satellites and on site sensors on land, air and sea to provide continuous information to the users \cite{www-ferro}. {\em Open Data Week} is an intergovernmental organization where 34 states come together to bring reforms and discuss how to use technology and services like {\em Copernicus} that use {\em Big Data} to test prototypes of new products to ensure they operate within the permissible levels of pollution \cite{www-ferro}. \\
While these initiatives help bring awareness about the seriousness of the issue, each state and country should take strict measures to bring out reforms that will help eradicate pollution. {\em Big Data} might never replace the environmental responsibility but it will help to plan the vision for environmental awareness and its tools make it easier to achieve the vision \cite{www-ferro}. These tools can also be used to gauge the alternative sources of energy and the feasibility of tapping into other natural resources ensuring responsible consumption of energy \cite{www-ferro}. For example, {\em IBM Bluemix} analyzed data from a steel industry and the analysis uncovered an interesting insight that the furnace wastes a lot of energy to offset the temperature of the smoke which resulted in optimizing its operation \cite{www-ibm1}

\section{Big Data Techniques to combat pollution}


\section{Conclusion}

    
\begin{acks}

The author would like to thank Dr. Gregor von Laszewski and the teaching assistants for their support and suggestions in writing this paper.

\end{acks}

\bibliographystyle{ACM-Reference-Format}
\bibliography{report} 

\end{document}
